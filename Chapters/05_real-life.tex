% !TEX root = ../Thesis.tex

\chapter{Progetti e applicazioni reali}\label{cap:05}

In questo capitolo verranno presentati progetti concreti che dimostrano
come Rust possa costituire una scelta valida non solo in termini teorici, ma anche nella pratica per lo sviluppo 
di basso livello quali driver, kernel e addirittura sistemi operativi completi.

I progetti analizzati sono in costante sviluppo ed evoluzione;
le informazioni riportate in questa trattazione sono aggiornate ad Agosto 2025, ma potrebbero subire
variazioni nel tempo.

Tra le applicazioni più rilevanti sono presenti: l'integrazione di Rust nel kernel di Windows 11,
il progetto \textit{Rust for Linux} (\textit{RfL}) con i relativi sottoprogetti, l'implementazione 
di \textit{sudo-rs} promossa da Ubuntu e il sistema operativo \textit{Redox OS}.\ 

\section{Kernel di Windows 11}
L'adozione di Rust da parte di Microsoft risale al 2023, quando l'azienda ha iniziato a riprogettare sezioni critiche del
kernel del sistema operativo Windows 11.

Prima di analizzare quest'adozione, è utile un breve excursus storico sull'evoluzione del kernel Windows, principalmente per comprendere l'importanza di questo cambiamento.
\begin{framed}
\noindent Lo sviluppo dei sistemi Microsoft iniziò con un kernel interamente scritto in assembly (\texttt{ASM8086}) nelle prime 
versioni di \textit{MS-DOS}.\ Con \textit{MS-DOS 3.0} venne introdotto il linguaggio C, ma il primo vero modulo kernel scritto in C
si ebbe con Windows 1.0 (\textbf{KERNEL.EXE}). Successivamente, C è rimasto il linguaggio principale per il kernel di Windows, con integrazioni C++ nelle
versioni successive.
\end{framed}
\noindent Il passaggio a Rust in Windows 11 è stato motivato da una combinazione di esigenze aziendali e del crescente interesse per il linguaggio Rust, che
iniziò a essere popolare proprio in questo periodo, grazie al modello offerto: garantire la sicurezza della memoria senza sacrificare le prestazioni.

In particolare, Rust ha permesso a Microsoft di affrontare due aspetti fondamentali:
\begin{itemize}
    \item \textbf{Sicurezza della memoria}: Windows ha una storia documentata alle spalle di errori legati alla memoria, quali \textit{null reference}, \textit{buffer overflow}, scritture illegali e altri. Il \textit{modello di ownership} offerto da Rust consente di prevenire queste problematiche già a tempo di compilazione;
    \item \textbf{Gestione della concorrenza}: Data la complessità Windows, anche gli errori di concorrenza sono frequenti. Il \textit{Borrow Checker} di Rust, insieme all'uso di \textit{smart pointers} per la condivisione dei dati, garantisce integrità e un uso corretto delle risorse condivise, eliminando errori quali \textit{race condition}.
\end{itemize}
Va sottolineato che, ad oggi, si tratta di un'adozione parziale: il kernel di Windows 11 è tuttora scritto prevalentemente in C, con alcune porzioni in C++.
L'obiettivo dichiarato di Microsoft è intervenire sulle sezioni chiave del kernel e sulle nuove funzionalità, valutando caso per caso l'impiego di Rust.

Nonostante si tratti di un'adozione parziale, questa scelta rappresenta un traguardo significativo per Rust: guadagnare la fiducia
di un colosso tecnologico come Microsoft ha contribuito alla crescita della popolarità del linguaggio, specialmente nel panorama della programmazione di sistema.

\section{Rust for Linux}

\section{Ubuntu: sudo-rs}

\section{Redox OS}