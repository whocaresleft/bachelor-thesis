% !TEX root = ../Thesis.tex

\chapter{Rust}\label{cap:02}

In questo capitolo verrà offerta una panoramica sul linguaggio di programmazione Rust, con particolare attenzione riguardo
le motivazioni che ne hanno determinato lo sviluppo e alla base della sua crescente popolarità.

Verranno inoltre introdotti alcuni concetti fondamentali riguardanti il modello di gestione della memoria di Rust, 
che costituiranno il fulcro del capitolo tre: `\textit{Gestione della memoria in Rust}' (cfr. Capitolo~\ref{cap:03}).

In chiusura, saranno presentati alcuni progetti attuali che evidenziano come Rust rappresenti una valida alternativa nel contesto della programmazione di sistemi.

Le informazioni contenute in questo capitolo sono tratte principalmente dal seguente articolo\cite{rust-2023-interview}.
\vspace{15pt}\\
\noindent Rust è un linguaggio di programmazione che, negli ultimi anni, ha ottenuto un'ottima reputazione  
tanto tra gli sviluppatori quanto tra le aziende. Il suo principale punto di forza risiede nella 
capacità, considerata a lungo un ideale solo teorico, di produrre codice che sia simultaneamente 
ad alte prestazioni e sicuro, in particolare sotto l'aspetto di gestione della memoria.

Che cosa rende Rust \textit{speciale}, rispetto ad altri linguaggi di programmazione? Principalmente, il suo approccio innovativo e alternativo alla gestione della memoria, che evita sia la necessità di gestione manuale da parte dello sviluppatore che la dipendenza da meccanismi di automatici, come \textit{garbage collection}.

\section{Contesto e motivazioni}
Rust è spesso definito come un linguaggio di basso livello 
che offre astrazioni di alto livello a costo quasi nullo.
La caratteristica principale che lo distingue dalla maggior parte degli altri linguaggi è il meccanismo
 di gestione della memoria dinamica. 

Tradizionalmente esistono due approcci principali alla gestione della memoria:
\begin{itemize}
    \item \textbf{Gestione manuale}: tipica di linguaggi considerati di livello più basso, come C e C++, progettati per dare al programmatore il massimo controllo sulle risorse, compresa la memoria. Secondo questo approccio è il programmatore a decidere quando allocare e quando deallocare la memoria dinamica. Il codice risultante generalmente è molto performante in termini di velocità d'esecuzione, ma questo avviene a un costo: non si hanno garanzie sulla sicurezza e sull'integrità della memoria. L'accesso e la gestione diretti della memoria possono portare problemi come \textit{unfreed memory}, \textit{access after free}, \textit{double free}, \textit{dangling pointers} e non solo.
    \item \textbf{Gestione automatica}: tipica di linguaggi di livello più alto come Java, Python e C\#, progettati per facilitare la gestione della memoria dinamica. È infatti gestita da un componente runtime, noto come \textit{Garbage Collector}: la memoria viene allocata implicitamente quando vengono creati oggetti, e liberata quando questi non sono più validi, in maniera trasparente agli occhi del programmatore. Il \textit{GC} tiene traccia degli oggetti allocati e libera la loro memoria quando questi non sono più referenziati. Il codice prodotto generalmente è libero dalla maggior parte degli errori legati alla memoria, ma questo avviene a un costo: il codice relativo al \textit{GC} viene incluso insieme al codice dell'applicazione, generando un file eseguibile di dimensioni maggiori; inoltre, il \textit{GC} richiede risorse computazionali aggiuntive per compiere la sua funzione, eventualmente rallentando, se non bloccando momentaneamente, l'esecuzione dell'applicazione.
\end{itemize}
Non esiste un approccio `oggettivamente corretto': durante lo sviluppo di un'applicazione 
la scelta sull'approccio utilizzato spesso ricade sulle specifiche esigenze.
In applicazioni di basso livello, quali lo sviluppo di sistemi operativi, dove le prestazioni sono fondamentali, la scelta si riduce, 
nella pratica, solamente all'approccio manuale, rendendo linguaggi come C e i suoi derivati lo standard.\hfill
\vspace{15pt}\\
Il creatore di Rust si pose come obiettivo sviluppare un linguaggio che eliminasse il compromesso tra prestazioni e sicurezza della memoria, fornendo un'alternativa agli approcci tradizionali.

\section{Origine e primo sviluppo}
Rust nacque inizialmente come progetto personale di Graydon Hoare, sviluppatore presso l'azienda Mozilla, nel 2006.
Il progetto nacque in seguito alla frustrazione di Hoare per i frequenti malfunzionamenti di software critici, spesso dovuti 
a problematiche di gestione della memoria\footnote{Secondo quanto riportato nell'articolo\cite{rust-2023-interview}, 
la frustrazione di Hoare era dovuta ai continui crash del software (presumibilmente scritto in C) dell'ascensore del palazzo in cui risiedeva, al ventunesimo piano.}.

L'obiettivo del progetto era molto ambizioso: sviluppare un linguaggio che consentisse di scrivere codice 
che fosse contemporaneamente \textit{veloce e sicuro}, principalmente sotto l'aspetto di gestione della memoria.

Il nome `\textit{Rust}' fu ispirato da un particolare tipo di fungo (chiamato appunto \textit{Rust}, ruggine in inglese), definito dal creatore 
stesso come `\textit{sovra-ingegnerizzati per la sopravvivenza}', metafora per la resilienza e l'attenzione 
alla sicurezza che il linguaggio mirava a ottenere. \hfill
\vspace{10pt}\\
\noindent Le prime versioni di Rust adottavano un sistema \textit{Garbage Collection} integrato nel linguaggio, dotato di una sintassi esplicita e dedicata.
Tuttavia, dopo i primi esperimenti, il team di sviluppo decise di rinunciare completamente al \textit{GC}, ritenendolo incompatibile con gli obiettivi di performance 
e controllo sulla memoria a cui mirava il linguaggio.

Fu così introdotto un nuovo modello, noto come `\textit{modello di ownership}', il quale si fonda 
sui concetti di possesso (\textit{ownership}) e prestito (\textit{borrowing}). 
Questo modello definisce regole rigorose sul trasferimento e l'utilizzo dei valori,cercando di prevenire gli errori legati 
alla memoria già durante la fase di compilazione, senza necessità di un \textit{runtime} dedicato (come il \textit{GC}). \hfill
\\
\noindent Questo modello divenne la base del sistema di memoria sicura senza \textit{GC} che ancora oggi contraddistingue Rust.

\section{Diffusione e adozione iniziali}
Dopo il rilascio della prima versione stabile, nel 2015, Rust iniziò a essere adottato in contesti reali, seppure con una
diffusione iniziale veramente limitata. Alcune aziende cominciarono a sperimentare con il linguaggio, per riscrivere
componenti critici, mirando a ottenere un miglioramento in termini di sicurezza e prestazioni:
\begin{itemize}
    \item \textbf{Mozilla}, promotrice del linguaggio, adottò Rust nella realizzazione del motore di rendering CSS \textit{Servo}, dal quale derivano alcuni componenti integrati nel motore di rendering di \textit{Firefox}, \textit{Gecko};
    \item \textbf{Facebook} riscrisse in Rust parte degli strumenti interni dedicati alla gestione della propria codebase.
\end{itemize}
Questi primi esempi segnarono una fase di `\textit{adozione esplorativa}', in cui Rust cominciava a essere percepito come una 
possibile alternativa ai linguaggi esistenti.

\section{Espansione e interesse crescenti}
Nel biennio 2020--2021 si verificò una prima vera e propria espansione dovuta a una maggiore stabilità del linguaggio. Le prime adozioni, 
da parte di Mozilla e Facebook, giocarono un ruolo cruciale: mostrarono le effettive capacità di Rust.

Come conseguenza, sempre più aziende cominciarono a introdurre codice Rust nella propria codebase, principalmente per i componenti critici sotto 
gli aspetti di prestazioni e sicurezza:
\begin{itemize}
    \item \textbf{Dropbox} migrò il motore di sincronizzazione da Python a Rust, ottenendo un significativo miglioramento delle prestazioni;
    \item \textbf{Fusion Engineering}\footnote{Si tratta di un'azienda che sviluppa sistemi software di controllo e gestione per droni. Fondata da Mara Bos, membro del \textit{Rust Project} e co-leader del \textit{Rust Library Team}.} utilizzò Rust per riscrivere i componenti relativi al sistema di controllo dei droni, precedentemente scritti in C++;
    \item \textbf{Discord} riscrisse la pipeline di encoding video e porzioni di backend relative alla gestione dell'autenticazione, dei messaggi e delle notifiche\footnote{Secondo dichiarazioni di Discord stessa, questa modifica ha generato un'incremento delle prestazioni fino a 10 volte.};
    \item \textbf{Amazon} iniziò a introdurre Rust nei servizi della piattaforma AWS, rafforzando il legame tra il linguaggio e l'ambiente cloud.
\end{itemize}
Tuttavia, in questo periodo, la maggior parte degli sviluppatori Rust contribuivano al progetto senza un supporto economico diretto.
I membri principali del team erano in gran parte impiegati in altre aziende (Mozilla, Amazon, Huawei e Microsoft) e si dedicavano a Rust come attività collaterale.

\section{Supporto istituzionale e professionale}
A partire dal 2021, Rust entrò in una nuova fase, contraddistinta da un crescente supporto industriale. Diverse aziende del settore 
tecnologico, in particolare alcune \textit{Big Tech}, iniziarono a finanziare in modo diretto lo sviluppo del linguaggio, offrendo compensi 
economici ai membri della community che divennero, a tutti gli effetti, sviluppatori Rust a tempo pieno.

L'obiettivo di questi finanziamenti era quello di permettere agli sviluppatori di dedicarsi in modo continuativo al progetto, garantendo
codice stabile, aggiornato e mantenibile. \hfill
\vspace{10pt}\\
\noindent Questo periodo segnò un passaggio cruciale: da un progetto guidato dalla passione della community a un ecosistema supportato anche
da risorse industriali. Di conseguenze, Rust cominciò a consolidarsi come una scelta sempre più pragmatica per applicazioni 
ad alte prestazioni e con forti requisiti in termini di sicurezza della memoria. \hfill
\vspace{15pt}\\
\noindent Negli ultimi anni, Rust ha trovato crescente impiego anche a livello di sistema operativo, in particolare in ambito kernel.
Tra gli esempi più rilevanti si trovano l'integrazione di Rust nel kernel di Windows 11 da parte di Microsoft,
il progetto \textit{Rust for Linux}, lo sviluppo di driver come \textit{Nova} da parte di Red Hat o il tentativo di rendere
\textit{sudo} più sicuro da parte di Ubuntu, fino ad arrivare allo sviluppo di interi sistemi operativi in Rust, come \textit{Redox OS}\footnote{Questi progetti verranno esplorati approfonditamente nel capitolo~\ref{cap:05}.}.