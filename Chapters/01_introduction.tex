% !TEX root = ../Thesis.tex

\chapter{Introduzione}\label{cap:01}
\section{Contestualizzazione}
Negli ultimi anni il linguaggio di programmazione \textit{Rust} ha suscitato 
un crescente interesse, attirando l'attenzione di sviluppatori e aziende.
Una delle motivazioni principali risiede nel suo approccio alla 
gestione della memoria dinamica, da sempre un tema cruciale nello sviluppo di sistemi di basso livello.

Storicamente, si sono affermati due approcci alla gestione della memoria: 
quello \textit{manuale}, adottato dai primi linguaggi di programmazione come 
l'assembly e, successivamente, dal C
e quello \textit{automatico}, introdotto nel 1959 da Lisp, il primo linguaggio 
a integrare la \textit{Garbage Collection} (\textit{GC}). \hfill 
\vspace{8pt}\\
\noindent L'approccio manuale rende il programmatore interamente responsabile 
della gestire della memoria, esponendo a una vasta gamma di errori dovuti al 
fattore umano; la \textit{GC}, invece, solleva il programmatore 
da tale incarico, ma introduce un overhead non trascurabile e riduce 
la trasparenza e il controllo sulla memoria.

Nel contesto delle applicazioni di basso livello, soprattutto dei sistemi 
operativi, controllo diretto e prestazioni sono requisiti fondamentali; 
per questo, l'approccio manuale rappresenta nella pratica l'unica soluzione percorribile.
È in questo contesto che si inserisce Rust, un linguaggio sviluppato per 
garantire una gestione sicura della memoria dinamica, senza impiegare \textit{GC}, e, 
quindi, senza compromettere le prestazioni. \hfill 
\vspace{8pt}\\
\noindent In questa tesi verrà analizzato Rust, con l'obiettivo di comprendere le 
motivazioni alla base della sua crescente diffusione e di valutare gli 
strumenti effettivamente offerti.

\section{Argomento}
La presente ricerca prende in esame la crescente popolarità di Rust nell'ambito
 della programmazione di basso livello, in particolare nello sviluppo di sistemi operativi.

 Il linguaggio ha una premessa chiara: prevenire interi classi di problemi legati
 alla gestione della memoria a livello di compilazione, 
fornendo allo stesso tempo prestazioni, in termini di velocità d'esecuzione, 
paragonabili a C e C++. \hfill
\vspace{8pt}\\
\noindent Vi sono, per tale motivo, varie iniziative che mirano a inserire il linguaggio
 come valida opzione nello sviluppo di sistemi operativi,
ma la sua integrazione solleva questioni fondamentali, sia da un punto di vista
 di tradizione (in quanto storicamente C è un pilastro nello sviluppo di sistemi
  operativi), che di spesa, in termini di tempo, 
 per imparare un nuovo linguaggio e per mantenere
  contemporaneamente due flussi di sviluppo paralleli (C e Rust).

\section{Obiettivi} 
L'obiettivo principale di questa tesi è offrire una panoramica sul linguaggio 
di programmazione Rust, esponendo quelle che sono le motivazioni 
dietro il suo sviluppo e la sua popolarità; successivamente, verrà analizzato 
l'approccio alternativo di gestione della memoria implementato dal linguaggio. \hfill
\vspace{7pt}\\
\noindent Un altro obiettivo è quello di stabilire se, nella teoria, Rust rappresenti una valida 
opzione per la programmazione di basso livello. Verranno innanzitutto stabiliti i requisiti 
fondamentali di un linguaggio per lo sviluppo di sistemi operativi; successivamente Rust verrà 
confrontato con C, standard \textit{de facto} per la programmazione di basso 
livello, sulla base di gestione delle risorse, della memoria e degli errori, complessità della sintassi e 
prestazioni, fornendo a tale scopo anche esempi di codice. \hfill
\vspace{7pt}\\
\noindent Come aspetto finale, verranno proposti progetti 
concreti che mostrano come la popolarità del linguaggio non sia solo 
teorica, ma rappresenti uno strumento pratico e in grado di portare risultati concreti.

L'analisi combina aspetti teorici (descrizione dei meccanismi ed esempi di codice)
 con lo studio di casi concreti, per evidenziare sia i principi 
alla base del linguaggio sia le sue applicazioni pratiche. 

\section{Struttura} 
Questa tesi è strutturata in quattro parti principali, ognuna con il rispettivo 
capitolo, dal due al cinque. \hfill
\vspace{7pt}\\
\noindent Nel Capitolo~\ref{cap:02}, verrà fornita una panoramica sul linguaggio di programmazione Rust, con particolare attenzione sulle motivazioni 
che ne hanno determinato lo svilupop e alla base della sua crescente popolarità negli ultimi anni. \hfill
\vspace{7pt}\\
\noindent Nel Capitolo~\ref{cap:03}, verrà esaminato il modello di gestione della memoria implementato da Rust,
con un'analisi approfondita dei meccanismi alla base; verranno inoltre forniti listati di codice che mostrano esempi del loro funzionamento. \hfill
\vspace{7pt}\\
\noindent Nel Capitolo~\ref{cap:04}, verranno esposti gli strumenti necessari che un linguaggio di programmazione dovrebbe avere per essere impiegato 
nella programmazione di sistema, con riferimento al linguaggio C. 
Successivamente, quest'ultimo verrà confrontato con Rust sotto gli aspetti di 
gestione della memoria, delle risorse e degli errori, complessità del codice e prestazioni, per mostrare come Rust metta a disposizione strumenti 
fondamentali per essere considerato una valida alternativa, teorica, nella programmazione di sistema. \hfill
\vspace{7pt}\\
\noindent Infine, nel Capitolo~\ref{cap:05}, verranno presentati progetti e applicazioni concreti che mostrano, nella pratica, come 
Rust non sia solo un linguaggio promettente dal punto di vista teorico, ma rappresenti un valido strumento, capace di ottenere risultati concreti, nella programmazione di basso livello.

\section{Annuncio dei risultati}
Nella prima parte della tesi (Capitoli~\ref{cap:02},~\ref{cap:03} e~\ref{cap:04}), viene evidenziato come l'approccio di gestione della memoria di Rust permetta 
di prevenire errori critici tipici dei linguaggi con approccio manuale, senza compromettere le prestazioni.

Il confronto con C evidenzia vantaggi in termini di sicurezza e mantenibilità, a fronte di una sintassi più complessa 
e una curva di apprendimento più ripida e lenta. \hfill 
\vspace{8pt}\\
\noindent Nella seconda parte (Capitolo~\ref{cap:05}), verranno presi in considerazioni progetti concreti (\textit{Rust for Linux}, \textit{Redox OS}, il kernel di Windows 11 e \texttt{sudo-rs}), 
evidenziando che Rust è già impiegato in contesti di sistemi operativi e che la sua adozione procede, seppur gradualmente, in maniera significativa. 