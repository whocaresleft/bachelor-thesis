% !TEX root = ../Thesis.tex

\chapter{Conclusione}\label{cap:06}
\section{Riassunto}
Il Capitolo~\ref{cap:02}, ha ripercorso le origini del linguaggio e le motivazioni alla base del suo sviluppo, 
evidenziando come si sia evoluto in breve tempo, da progetto personale e sperimentale, in uno strumento adottato da un ampio numero di sviluppatori e aziende. \hfill
\vspace{10pt}\\
\noindent Nel Capitolo~\ref{cap:03}, sono stati analizzati a fondo i meccanismi di \textit{ownership}, \textit{borrowing} e \textit{lifetime}, che costituiscono 
il fulcro del modello di gestione della memoria del linguaggio. 
Questi strumenti hanno mostrato come Rust riesce a garantire simultaneamente sicurezza e controllo, aspetti
fondamentali per lo sviluppo di basso livello. \hfill
\vspace{10pt}\\
\noindent Il Capitolo~\ref{cap:04}, ha preso in analisi il linguaggio C come riferimento
storico e pratico, ricostruendo le caratteristiche che lo hanno reso centrale nella programmazione di sistema: 
\textit{compilazione}, \textit{assenza di runtime}, \textit{manipolazione della memoria} e dei \textit{bit}.
Rust è stato successivamente confrontato con C sia negli aspetti precedentemente elencati, sia in 
aspetti trasversali, quali \textit{gestione delle risorse}, \textit{sicurezza della memoria}, \textit{complessità della sintassi} e \textit{prestazioni}.

Ne è emerso un quadro in cui Rust rappresenta una valida alternativa, nella teoria, a C, 
a costo di una curva di apprendimento più ripida rispetto a quest'ultimo. \hfill
\vspace{10pt}\\
\noindent Infine, il Capitolo~\ref{cap:05}, ha illustrato progetti
concreti già in corso che sperimentano l'uso del linguaggio nello sviluppo di sistemi operativi e 
componenti critici. Queste iniziative mostrano come il linguaggio non sia più una 
premessa puramente teorica, ma uno strumento concreto già in grado di produrre risultati significativi. \hfill
\vspace{10pt}\\
\noindent Nel complesso, l'analisi ha messo alla luce le potenzialità e i limiti di Rust nello sviluppo di 
basso livello, aspetti che verranno approfonditi nella prossima sezione, `\textit{Considerazioni critiche e personali}'~\ref{sec:critique}.

\section{Considerazioni critiche e personali}\label{sec:critique}
Rust appare in grado di portare benefici significativi nello sviluppo di applicazioni di basso livello e, in particolare, di sistemi operativi.
Grazie al \textit{modello di ownership}, gli errori di gestione della memoria dinamica vengono di fatto eliminati a tempo di compilazione; inoltre 
le astrazioni \textit{zero-cost} e l'impiego di \textit{smart pointer} consentono una gestione sicura della concorrenza, prevenendo interamente le \textit{data race}.

Tuttavia, Rust non rappresenta una soluzione magica né priva di limitazioni. La sintassi e il \textit{Borrow Checker} richiedono tempo e dedizione per essere
padroneggiati, definendo una curva di apprendimento molto più ripida rispetto a linguaggi più permissivi come Python o anche C. A questo va aggiunto il fatto 
che Rust si limita a prevenire gli errori di gestione della memoria, ma non quelli logici: la correttezza di un algoritmo o di un'implementazione 
rimane ancora pienamente responsabilità del programmatore. \hfill 
\vspace{10pt}\\
\noindent È quindi opportuno, o comunque consigliato, evitare un impiego indiscriminato del linguaggio. Da un lato, utilizzare Rust in contesti 
dove i suoi vantaggi non si traducono in benefici concreti 
(ad esempio, \  un'applicazione che non lavora con una quantità significativa di memoria dinamica)
rischia di introdurre soltanto complessità;
dall'altro, utilizzare Rust come se fosse un'altro linguaggio, aggirando o utilizzando in maniera scorretta i costrutti 
disponibili (per esempio scrivendo ampie porzioni di codice \textit{unsafe}) vanifica gran parte dei 
principi alla base del linguaggio stesso. \hfill 
\vspace{10pt}\\
\noindent La prospettiva più equilibrata è dunque quella di impiegare Rust nei contesti critici, in cui sia le prestazioni che la sicurezza della memoria sono 
fondamentali e in cui i vantaggi offerti si traducono in maniera concreta, evitando un uso eccessivo o improprio che ne comprometterebbe i punti di forza.