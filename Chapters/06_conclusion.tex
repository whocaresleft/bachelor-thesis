% !TEX root = ../Thesis.tex

\chapter{Conclusione}\label{cap:06}
\section{Riassunto}
Il secondo capitolo, `\textit{Rust}'~\ref{cap:02}, ha ripercorso le origini del linguaggio e le motivazioni alla base del suo sviluppo, 
evidenziando come si sia evoluto in breve tempo, da progetto personale e sperimentale, in uno strumento adottato da un ampio numero di sviluppatori e aziende. \hfill
\vspace{10pt}\\
\noindent Nel terzo capitolo, `\textit{Gestione della memoria in Rust}'~\ref{cap:03}, sono stati analizzati a fondo i meccanismi di \textit{ownership}, \textit{borrowing} e \textit{lifetime}, che costituiscono 
il fulcro del modello di gestione della memoria del linguaggio. 
Questi strumenti hanno mostrato come Rust riesce a garantire simultaneamente sicurezza e controllo, aspetti
fondamentali per lo sviluppo di basso livello. \hfill
\vspace{10pt}\\
\noindent Il quarto capitolo, `\textit{Sistemi Operativi}'~\ref{cap:04}, ha preso in analisi il linguaggio C come riferimento
storico e pratico, ricostruendo le caratteristiche che lo hanno reso centrale nella programmazione di sistema: 
\textit{compilazione}, \textit{assenza di runtime}, \textit{manipolazione della memoria} e dei \textit{bit}.
Rust è stato successivamente confrontato con C sia negli aspetti precedentemente elencati, sia in 
aspetti trasversali, quali \textit{gestione delle risorse}, \textit{sicurezza della memoria}, \textit{complessità della sintassi} e \textit{prestazioni}.

Ne è emerso un quadro in cui Rust rappresenta una valida alternativa, nella teoria, a C, 
a costo di una curva di apprendimento più ripida rispetto a quest'ultimo. \hfill
\vspace{10pt}\\
\noindent Infine, il quinto capitolo, `\textit{Progetti e applicazioni reali}'~\ref{cap:05}, ha illustrato progetti
concreti già in corso che sperimentano l'uso del linguaggio nello sviluppo di sistemi operativi e 
componenti critici. Queste iniziative mostrano come il linguaggio non sia più una 
premessa puramente teorica, ma uno strumento concreto già in grado di produrre risultati significativi. \hfill
\vspace{10pt}\\
\noindent Nel complesso, l'analisi ha messo alla luce le potenzialità e i limiti di Rust nello sviluppo di 
basso livello, aspetti che verranno approfonditi nella prossima sezione, `\textit{Considerazioni critiche e personali}'~\ref{sec:critique}.

\section{Considerazioni critiche e personali}\label{sec:critique}