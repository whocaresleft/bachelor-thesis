\documentclass{article}
\usepackage[utf8]{inputenc}
\begin{document}

\paragraph{Apertura}
[Buon giorno / Buona sera], egregi esaminatori, professori, colleghi e chiunque sia presente qua oggi.

\noindent [Considerando che sono il primo candidato, posso sperare di avere tutta la vostra attenzione.
 Considerando di essere l'ultimo candidato, spero di chiudere per tutti la giornata in bellezza.
 / Prima di iniziare, voglio cogliere questo momento per ringraziare tutti coloro che sono presenti per la mia presentazione.]

\noindent Mi presento, sono Francesco Biribo, laureando al corso triennale in Informatica e la mia tesi si intitola:
`\textit{Analisi del linguaggio Rust e della sua adozione nei sistemi operativi}'. \hfill
\vspace{5pt}\\
\noindent Sicuramente quasi a tutti sara capitato, mentre utilizzano un PC, di assistere ad applicazioni,
o peggio ancora, al sistema operativo, di bloccarsi o crashare del tutto. Non sempre, ma in molti casi questi 
problemi sono dovuti a errori legati alla memoria, o meglio, a una sua cattiva o inefficiente gestione (su questo ci torneremo a breve).
Il mio lavoro di tesi parte proprio da questo contesto. \hfill
\vspace{5pt}\\
\noindent Con questa presentazione intendo fornire un contesto generale sul linguaggio di programmazione Rust, 
introducendo i suoi aspetti chiave e le motivazioni dietro alla sua crescente adozione in svariati progetti,
in particolare quelli di basso livello (come i sistemi operativi), facendo riferimento anche ad alcuni di essi;
infine, forniro qualche consiglio generale riguardo al linguaggio.



\end{document}